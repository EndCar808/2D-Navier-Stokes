%---------------------------PREAMBLE-------------------------------------------- %
\documentclass[9pt]{article}
% \documentclass[9pt]{amsart}
% \documentclass[9pt]{revtex4-2}


%------------------- PACKAGES 

\usepackage{graphicx}
	\graphicspath{ {plots/} }
\usepackage{caption}
\usepackage{epstopdf}
\usepackage{pdfpages}
\usepackage{array}
\usepackage{ulem}
\usepackage{amsfonts}
\usepackage{amssymb}
\usepackage{amsmath}
\usepackage{amsthm}
	\newtheorem{definition}{Definition}[section]
	\newtheorem*{remark}{Remark}
	\newtheorem{theorem}{Theorem}[section]
	\newtheorem{corollary}{Corollary}[theorem]
	\newtheorem{lemma}[theorem]{Lemma}
\usepackage{mathrsfs}
\usepackage{color}
\usepackage{float}
\usepackage{tikz}
\usetikzlibrary{arrows,%
                plotmarks,patterns}
\usepackage{pgfplots}
\pgfplotsset{compat=newest}
\usepgfplotslibrary{fillbetween}

\usepackage{enumerate}
\usepackage{array}
\usepackage{wrapfig}
\usepackage{enumitem}
\usepackage{bm}
\usepackage{booktabs}
\usepackage{siunitx}
\usepackage{authblk}
\usepackage{cancel}
\usepackage{listings}
\usepackage[utf8]{inputenc}

% \usepackage[table]{xcolor}

\usepackage[unicode,psdextra, colorlinks=true, linkcolor=black, citecolor=black, urlcolor=blue, breaklinks]{hyperref}[2012/08/13]


\setlength{\arrayrulewidth}{0.5mm}
% \setlength{\tabcolsep}{10pt}
\renewcommand{\arraystretch}{1.0}


%------------------ USEFULL MACROS ------------------%
%-- MISC
\newcommand{\comment}[1]{}                                % for adding an inline comment
\makeatletter 											  % This entire thing is for redefining *matrix environment in amsmath so that you can specify the line spacing in a matrix by  - \begin{pmatrix}[1.5]
\renewcommand*\env@matrix[1][\arraystretch]{%             
  \edef\arraystretch{#1}%
  \hskip -\arraycolsep
  \let\@ifnextchar\new@ifnextchar
  \array{*\c@MaxMatrixCols c}}
\makeatother

%-- DERIVATIVES
\newcommand{\der}[2]{\frac{\mathrm{d}#1}{\mathrm{d}#2}}          	 % first derivative
\newcommand{\dder}[2]{\frac{\mathrm{d}^{2}#1}{\mathrm{d}#2^{2}}}	 % second derivative
\newcommand{\nder}[3]{\frac{\mathrm{d}^{#3}#1}{\mathrm{d}#2^{#3}}}   % nth derivative
\newcommand{\pder}[2]{\frac{\partial #1}{\partial #2}}               % first partial derivative
\newcommand{\ppder}[2]{\frac{\partial^2 #1}{\partial {#2}^2}}        % second partial derivative
\newcommand{\npder}[3]{\frac{\partial^{#3} #1}{\partial {#2}^{#3}}}  % second partial derivative

%-- SUMMATION
\newcommand{\sumk}[3]{\sum_{#1 = #2}^{#3}}        			% sum over #1 with limits #2 #3
\newcommand{\sumkk}[2]{\sum_{#1, #2} }         	  			% sum over #1 and #2 no limits
\newcommand{\sumkkdom}[3]{\sum_{#1 , #2 \in #3}}   			% sum over #1 and #2 with domain specified

%-- RESEARCH RELATED
\newcommand{\uhat}[1]{\hat{u}_{#1}}      				          % quick fourier mode
\newcommand{\akak}[2]{a_{#1}a_{#2}}      				          % quick convolution amplitudes
\newcommand{\akakak}[3]{\frac{a_{#1}a_{#2}}{a_{#3}}}      	      % quick convolution amplitudes
\newcommand{\triadexpl}[3]{\phi_{#1} + \phi_{#2} - \phi_{#3}}      % quick triad explicitly
\newcommand{\triad}[3]{\varphi_{#1, #2}^{#3}}                     % quick varphi triad
\newcommand{\ii}{\mathrm{i}}      								  % imaginary i
\newcommand{\e}{\mathrm{e}}      								  % e
\newcommand{\Mod}[1]{\ (\mathrm{mod}\ #1)}
\newcommand{\grad}[1]{\nabla{#1}}								% gradient operator
\newcommand{\curl}[1]{\nabla \times {#1}}								% gradient operator
\newcommand{\diverg}[1]{\nabla \cdot {#1}}			% divergence operator
\newcommand{\bfu}{\mathbf{u}}											% vector u in bold font
\newcommand{\omegahat}[1]{\hat{\omega}_{ \mathbf{#1} } }								% gradient operator
\newcommand{\alphakkk}[3]{\alpha_{\bfkn{#1}, \bfkn{#2}}^{\bfkn{#3}}}
\newcommand{\bfx}{\mathbf{x}}								% gradient operator
\newcommand{\bfk}{\mathbf{k}}								% gradient operator
\newcommand{\bfkn}[1]{\mathbf{k}_{#1}}								% gradient operator

%-- COMMENTS / EDITING
\newcommand{\TODO}[1]{\textcolor{red}{TODO: #1}}


%-- CODE SNIPPETS
\newcommand{\code}[1]{\texttt{#1}}





%-------- PAGE STYLE & MARGIN SIZE
% \pagestyle{plain} \setlength{\oddsidemargin}{0.05in}
% \setlength{\evensidemargin}{0.05in} \setlength{\topmargin}{0in}
% \setlength{\footskip}{1.05in} \setlength{\headsep}{0in}
% \setlength{\textwidth}{6.4in} \setlength{\textheight}{8.25in}
\usepackage[a4paper, margin=0.5cm]{geometry}




\title{\textbf{Phase Dynamics - 2D Navier Stokes}}

\author[$1$]{E. M. Carroll}
\author[$1$]{M. D. Bustamante}
\affil[$1$]{Department of Mathematics and Statistics, University College Dublin, Dublin, Ireland} 





\begin{document}


\maketitle	


\section{Enstorphy Flux}

The flux of enstrophy in/out of $\mathcal{C}_{\theta}$ is defined as

\begin{align}
\Pi_{\mathcal{C}_{\theta}} &= 2 \pi^2 \sum_{\substack{\bfkn{3} \in \mathcal{C}_{\theta} \\ \bfkn{1},  \bfkn{2} \in \mathcal{C}_{\theta}^{'} \\ \bfkn{1} + \bfkn{2} = \bfkn{3}}} \left(k_{1 x} k_{2 y}-k_{2 x} k_{1 y}\right)\left(\frac{1}{\left|\mathbf{k}_{1}\right|^{2}}-\frac{1}{\left|\mathbf{k}_{2}\right|^{2}}\right) a_{\mathbf{k}_{1}} a_{\mathbf{k}_{2}} a_{\mathbf{k}_{3}} \cos \left(\varphi_{\mathbf{k}_{1} \mathbf{k}_{2}}^{\mathbf{k}_{3}}\right) \\
&\qquad - 2 \pi^2 \sum_{\substack{\bfkn{1}, \bfkn{2} \in \mathcal{C}_{\theta} \\ \bfkn{3} \in \mathcal{C}_{\theta}^{'} \\ \bfkn{1} + \bfkn{2} = \bfkn{3}}} \left(k_{1 x} k_{2 y}-k_{2 x} k_{1 y}\right)\left(\frac{1}{\left|\mathbf{k}_{1}\right|^{2}}-\frac{1}{\left|\mathbf{k}_{2}\right|^{2}}\right) a_{\mathbf{k}_{1}} a_{\mathbf{k}_{2}} a_{\mathbf{k}_{3}} \cos \left(\varphi_{\mathbf{k}_{1} \mathbf{k}_{2}}^{\mathbf{k}_{3}}\right)
\end{align}

\subsection{Partition of Wavevector Space}


% Pattern Info
 
\tikzset{
pattern size/.store in=\mcSize, 
pattern size = 5pt,
pattern thickness/.store in=\mcThickness, 
pattern thickness = 0.3pt,
pattern radius/.store in=\mcRadius, 
pattern radius = 1pt}\makeatletter
\pgfutil@ifundefined{pgf@pattern@name@_9loxjy35x}{
\pgfdeclarepatternformonly[\mcThickness,\mcSize]{_9loxjy35x}
{\pgfqpoint{-\mcThickness}{-\mcThickness}}
{\pgfpoint{\mcSize}{\mcSize}}
{\pgfpoint{\mcSize}{\mcSize}}
{\pgfsetcolor{\tikz@pattern@color}
\pgfsetlinewidth{\mcThickness}
\pgfpathmoveto{\pgfpointorigin}
\pgfpathlineto{\pgfpoint{\mcSize}{0}}
\pgfpathmoveto{\pgfpointorigin}
\pgfpathlineto{\pgfpoint{0}{\mcSize}}
\pgfusepath{stroke}}}
\makeatother

% Pattern Info
 
\tikzset{
pattern size/.store in=\mcSize, 
pattern size = 5pt,
pattern thickness/.store in=\mcThickness, 
pattern thickness = 0.3pt,
pattern radius/.store in=\mcRadius, 
pattern radius = 1pt}
\makeatletter
\pgfutil@ifundefined{pgf@pattern@name@_bse86zqf7}{
\pgfdeclarepatternformonly[\mcThickness,\mcSize]{_bse86zqf7}
{\pgfqpoint{0pt}{0pt}}
{\pgfpoint{\mcSize+\mcThickness}{\mcSize+\mcThickness}}
{\pgfpoint{\mcSize}{\mcSize}}
{
\pgfsetcolor{\tikz@pattern@color}
\pgfsetlinewidth{\mcThickness}
\pgfpathmoveto{\pgfqpoint{0pt}{0pt}}
\pgfpathlineto{\pgfpoint{\mcSize+\mcThickness}{\mcSize+\mcThickness}}
\pgfusepath{stroke}
}}
\makeatother

% Pattern Info
 
\tikzset{
pattern size/.store in=\mcSize, 
pattern size = 5pt,
pattern thickness/.store in=\mcThickness, 
pattern thickness = 0.3pt,
pattern radius/.store in=\mcRadius, 
pattern radius = 1pt}
\makeatletter
\pgfutil@ifundefined{pgf@pattern@name@_imlgqtpiq}{
\pgfdeclarepatternformonly[\mcThickness,\mcSize]{_imlgqtpiq}
{\pgfqpoint{0pt}{0pt}}
{\pgfpoint{\mcSize+\mcThickness}{\mcSize+\mcThickness}}
{\pgfpoint{\mcSize}{\mcSize}}
{
\pgfsetcolor{\tikz@pattern@color}
\pgfsetlinewidth{\mcThickness}
\pgfpathmoveto{\pgfqpoint{0pt}{0pt}}
\pgfpathlineto{\pgfpoint{\mcSize+\mcThickness}{\mcSize+\mcThickness}}
\pgfusepath{stroke}
}}
\makeatother

% Pattern Info
 
\tikzset{
pattern size/.store in=\mcSize, 
pattern size = 5pt,
pattern thickness/.store in=\mcThickness, 
pattern thickness = 0.3pt,
pattern radius/.store in=\mcRadius, 
pattern radius = 1pt}
\makeatletter
\pgfutil@ifundefined{pgf@pattern@name@_iqb6bmvtb}{
\pgfdeclarepatternformonly[\mcThickness,\mcSize]{_iqb6bmvtb}
{\pgfqpoint{0pt}{0pt}}
{\pgfpoint{\mcSize+\mcThickness}{\mcSize+\mcThickness}}
{\pgfpoint{\mcSize}{\mcSize}}
{
\pgfsetcolor{\tikz@pattern@color}
\pgfsetlinewidth{\mcThickness}
\pgfpathmoveto{\pgfqpoint{0pt}{0pt}}
\pgfpathlineto{\pgfpoint{\mcSize+\mcThickness}{\mcSize+\mcThickness}}
\pgfusepath{stroke}
}}
\makeatother

% Pattern Info
 
\tikzset{
pattern size/.store in=\mcSize, 
pattern size = 5pt,
pattern thickness/.store in=\mcThickness, 
pattern thickness = 0.3pt,
pattern radius/.store in=\mcRadius, 
pattern radius = 1pt}
\makeatletter
\pgfutil@ifundefined{pgf@pattern@name@_o7glz41qa}{
\pgfdeclarepatternformonly[\mcThickness,\mcSize]{_o7glz41qa}
{\pgfqpoint{0pt}{0pt}}
{\pgfpoint{\mcSize+\mcThickness}{\mcSize+\mcThickness}}
{\pgfpoint{\mcSize}{\mcSize}}
{
\pgfsetcolor{\tikz@pattern@color}
\pgfsetlinewidth{\mcThickness}
\pgfpathmoveto{\pgfqpoint{0pt}{0pt}}
\pgfpathlineto{\pgfpoint{\mcSize+\mcThickness}{\mcSize+\mcThickness}}
\pgfusepath{stroke}
}}
\makeatother

% Pattern Info
 
\tikzset{
pattern size/.store in=\mcSize, 
pattern size = 5pt,
pattern thickness/.store in=\mcThickness, 
pattern thickness = 0.3pt,
pattern radius/.store in=\mcRadius, 
pattern radius = 1pt}
\makeatletter
\pgfutil@ifundefined{pgf@pattern@name@_pfm9ibd2v}{
\pgfdeclarepatternformonly[\mcThickness,\mcSize]{_pfm9ibd2v}
{\pgfqpoint{0pt}{-\mcThickness}}
{\pgfpoint{\mcSize}{\mcSize}}
{\pgfpoint{\mcSize}{\mcSize}}
{
\pgfsetcolor{\tikz@pattern@color}
\pgfsetlinewidth{\mcThickness}
\pgfpathmoveto{\pgfqpoint{0pt}{\mcSize}}
\pgfpathlineto{\pgfpoint{\mcSize+\mcThickness}{-\mcThickness}}
\pgfusepath{stroke}
}}
\makeatother

% Pattern Info
 
\tikzset{
pattern size/.store in=\mcSize, 
pattern size = 5pt,
pattern thickness/.store in=\mcThickness, 
pattern thickness = 0.3pt,
pattern radius/.store in=\mcRadius, 
pattern radius = 1pt}
\makeatletter
\pgfutil@ifundefined{pgf@pattern@name@_2hninm6s4}{
\pgfdeclarepatternformonly[\mcThickness,\mcSize]{_2hninm6s4}
{\pgfqpoint{0pt}{-\mcThickness}}
{\pgfpoint{\mcSize}{\mcSize}}
{\pgfpoint{\mcSize}{\mcSize}}
{
\pgfsetcolor{\tikz@pattern@color}
\pgfsetlinewidth{\mcThickness}
\pgfpathmoveto{\pgfqpoint{0pt}{\mcSize}}
\pgfpathlineto{\pgfpoint{\mcSize+\mcThickness}{-\mcThickness}}
\pgfusepath{stroke}
}}
\makeatother
\tikzset{every picture/.style={line width=0.75pt}} %set default line width to 0.75pt      

  

\begin{figure}[H]
\centering
  \begin{tikzpicture}[x=0.75pt,y=0.75pt,yscale=-1,xscale=1]
    %uncomment if require: \path (0,445); %set diagram left start at 0, and has height of 445

    %Shape: Circle [id:dp8047077122910686] 
    \draw  [color={rgb, 255:red, 184; green, 233; blue, 134 }  ,draw opacity=1 ][pattern=_9loxjy35x,pattern size=6pt,pattern thickness=0.75pt,pattern radius=0pt, pattern color={rgb, 255:red, 184; green, 233; blue, 134}][line width=1.5]  (180.75,197) .. controls (180.75,150.58) and (218.38,112.95) .. (264.8,112.95) .. controls (311.22,112.95) and (348.85,150.58) .. (348.85,197) .. controls (348.85,243.42) and (311.22,281.05) .. (264.8,281.05) .. controls (218.38,281.05) and (180.75,243.42) .. (180.75,197) -- cycle ;
    %Shape: Axis 2D [id:dp0010428756563580777] 
    \draw [color={rgb, 255:red, 155; green, 155; blue, 155 }  ,draw opacity=1 ] (440.99,196.25) -- (655.99,196.25)(546.79,97.25) -- (546.79,291.25) (648.99,191.25) -- (655.99,196.25) -- (648.99,201.25) (541.79,104.25) -- (546.79,97.25) -- (551.79,104.25)  ;
    %Shape: Circle [id:dp01781541677683296] 
    \draw  [color={rgb, 255:red, 155; green, 155; blue, 155 }  ,draw opacity=1 ][dash pattern={on 4.5pt off 4.5pt}] (462.75,196) .. controls (462.75,149.58) and (500.38,111.95) .. (546.8,111.95) .. controls (593.22,111.95) and (630.85,149.58) .. (630.85,196) .. controls (630.85,242.42) and (593.22,280.05) .. (546.8,280.05) .. controls (500.38,280.05) and (462.75,242.42) .. (462.75,196) -- cycle ;
    %Shape: Pie [id:dp2605199446590514] 
    \draw  [color={rgb, 255:red, 74; green, 144; blue, 226 }  ,draw opacity=1 ][dash pattern={on 5.63pt off 4.5pt}][line width=1.5]  (496.21,263.13) .. controls (484.7,254.44) and (475.49,242.85) .. (469.67,229.44) -- (546.8,196) -- cycle ;
    %Shape: Pie [id:dp13165810013367563] 
    \draw  [color={rgb, 255:red, 74; green, 144; blue, 226 }  ,draw opacity=1 ][dash pattern={on 5.63pt off 4.5pt}][line width=1.5]  (597.88,129.24) .. controls (608.46,137.35) and (617.05,147.92) .. (622.81,160.09) -- (546.8,196) -- cycle ;
    %Shape: Block Arc [id:dp0537731675897255] 
    \draw  [color={rgb, 255:red, 74; green, 144; blue, 226 }  ,draw opacity=1 ][pattern=_bse86zqf7,pattern size=6pt,pattern thickness=0.75pt,pattern radius=0pt, pattern color={rgb, 255:red, 74; green, 144; blue, 226}][line width=1.5]  (597.85,129.42) .. controls (608.43,137.62) and (616.99,148.31) .. (622.67,160.6) -- (606.17,168.3) .. controls (601.73,158.67) and (595.04,150.3) .. (586.78,143.87) -- cycle ;
    %Shape: Block Arc [id:dp6844600508043197] 
    \draw  [color={rgb, 255:red, 74; green, 144; blue, 226 }  ,draw opacity=1 ][pattern=_imlgqtpiq,pattern size=6pt,pattern thickness=0.75pt,pattern radius=0pt, pattern color={rgb, 255:red, 74; green, 144; blue, 226}][line width=1.5]  (496.21,263.13) .. controls (484.75,254.48) and (475.56,242.96) .. (469.74,229.64) -- (486.58,222.28) .. controls (491.13,232.69) and (498.31,241.69) .. (507.27,248.45) -- cycle ;
    %Curve Lines [id:da7878239924028905] 
    \draw [color={rgb, 255:red, 0; green, 0; blue, 0 }  ,draw opacity=1 ]   (424.6,249.3) .. controls (464.2,228.51) and (447.35,272.11) .. (485.82,243.88) ;
    \draw [shift={(487,243)}, rotate = 143.13] [color={rgb, 255:red, 0; green, 0; blue, 0 }  ,draw opacity=1 ][line width=0.75]    (10.93,-3.29) .. controls (6.95,-1.4) and (3.31,-0.3) .. (0,0) .. controls (3.31,0.3) and (6.95,1.4) .. (10.93,3.29)   ;
    %Curve Lines [id:da6221125644337466] 
    \draw    (677.6,135.3) .. controls (669.06,129.75) and (662.88,126.93) .. (658.07,125.99) .. controls (640.73,122.6) and (641.12,143.65) .. (611.83,148.71) ;
    \draw [shift={(610,149)}, rotate = 351.81] [color={rgb, 255:red, 0; green, 0; blue, 0 }  ][line width=0.75]    (10.93,-3.29) .. controls (6.95,-1.4) and (3.31,-0.3) .. (0,0) .. controls (3.31,0.3) and (6.95,1.4) .. (10.93,3.29)   ;
    %Curve Lines [id:da6960340400668596] 
    \draw    (478.6,185.3) .. controls (501.25,235.53) and (500.82,182.64) .. (505.77,224.04) ;
    \draw [shift={(506,226)}, rotate = 263.41] [color={rgb, 255:red, 0; green, 0; blue, 0 }  ][line width=0.75]    (10.93,-3.29) .. controls (6.95,-1.4) and (3.31,-0.3) .. (0,0) .. controls (3.31,0.3) and (6.95,1.4) .. (10.93,3.29)   ;
    %Curve Lines [id:da865163833025945] 
    \draw    (594.6,213.3) .. controls (606.48,178.65) and (566.61,211.33) .. (588.91,163.77) ;
    \draw [shift={(589.6,162.3)}, rotate = 115.59] [color={rgb, 255:red, 0; green, 0; blue, 0 }  ][line width=0.75]    (10.93,-3.29) .. controls (6.95,-1.4) and (3.31,-0.3) .. (0,0) .. controls (3.31,0.3) and (6.95,1.4) .. (10.93,3.29)   ;
    %Shape: Axis 2D [id:dp1759775737443261] 
    \draw [color={rgb, 255:red, 155; green, 155; blue, 155 }  ,draw opacity=1 ] (159,197) -- (374,197)(264.8,98) -- (264.8,292) (367,192) -- (374,197) -- (367,202) (259.8,105) -- (264.8,98) -- (269.8,105)  ;
    %Shape: Block Arc [id:dp26518686054278096] 
    \draw  [color={rgb, 255:red, 255; green, 255; blue, 255 }  ,draw opacity=1 ][fill={rgb, 255:red, 255; green, 255; blue, 255 }  ,fill opacity=1 ][line width=1.5]  (315.2,129.92) .. controls (325.86,138.02) and (334.52,148.63) .. (340.33,160.86) -- (323.9,168.72) .. controls (319.36,159.14) and (312.59,150.83) .. (304.26,144.48) -- cycle ;
    %Shape: Block Arc [id:dp9572201801422198] 
    \draw  [color={rgb, 255:red, 255; green, 255; blue, 255 }  ,draw opacity=1 ][fill={rgb, 255:red, 255; green, 255; blue, 255 }  ,fill opacity=1 ][line width=1.5]  (214.24,264.15) .. controls (202.77,255.5) and (193.59,243.98) .. (187.76,230.66) -- (204.6,223.3) .. controls (209.16,233.71) and (216.33,242.71) .. (225.29,249.47) -- cycle ;
    %Shape: Block Arc [id:dp9816655274702781] 
    \draw  [color={rgb, 255:red, 74; green, 144; blue, 226 }  ,draw opacity=1 ][pattern=_iqb6bmvtb,pattern size=6pt,pattern thickness=0.75pt,pattern radius=0pt, pattern color={rgb, 255:red, 74; green, 144; blue, 226}][line width=1.5]  (214.24,264.15) .. controls (202.77,255.5) and (193.59,243.98) .. (187.76,230.66) -- (204.6,223.3) .. controls (209.16,233.71) and (216.33,242.71) .. (225.29,249.47) -- cycle ;
    %Shape: Block Arc [id:dp1787138542850888] 
    \draw  [color={rgb, 255:red, 74; green, 144; blue, 226 }  ,draw opacity=1 ][pattern=_o7glz41qa,pattern size=6pt,pattern thickness=0.75pt,pattern radius=0pt, pattern color={rgb, 255:red, 74; green, 144; blue, 226}][line width=1.5]  (315.2,129.92) .. controls (325.86,138.02) and (334.52,148.63) .. (340.33,160.86) -- (323.9,168.72) .. controls (319.36,159.14) and (312.59,150.83) .. (304.26,144.48) -- cycle ;
    %Curve Lines [id:da2625008465286751] 
    \draw    (165.6,123.3) .. controls (138.73,147.18) and (265.32,189.87) .. (204.92,239.26) ;
    \draw [shift={(204,240)}, rotate = 321.55] [color={rgb, 255:red, 0; green, 0; blue, 0 }  ][line width=0.75]    (10.93,-3.29) .. controls (6.95,-1.4) and (3.31,-0.3) .. (0,0) .. controls (3.31,0.3) and (6.95,1.4) .. (10.93,3.29)   ;
    %Curve Lines [id:da4624334379758528] 
    \draw    (196.6,104.3) .. controls (236.4,74.45) and (254.42,199.04) .. (321.98,150.74) ;
    \draw [shift={(323,150)}, rotate = 143.67] [color={rgb, 255:red, 0; green, 0; blue, 0 }  ][line width=0.75]    (10.93,-3.29) .. controls (6.95,-1.4) and (3.31,-0.3) .. (0,0) .. controls (3.31,0.3) and (6.95,1.4) .. (10.93,3.29)   ;
    %Curve Lines [id:da9434944999097219] 
    \draw    (310.6,292.3) .. controls (350.4,262.45) and (255.56,289.03) .. (293.42,223) ;
    \draw [shift={(294,222)}, rotate = 120.35] [color={rgb, 255:red, 0; green, 0; blue, 0 }  ][line width=0.75]    (10.93,-3.29) .. controls (6.95,-1.4) and (3.31,-0.3) .. (0,0) .. controls (3.31,0.3) and (6.95,1.4) .. (10.93,3.29)   ;
    %Shape: Pie [id:dp3666734869844357] 
    \draw  [color={rgb, 255:red, 208; green, 2; blue, 27 }  ,draw opacity=1 ][pattern=_pfm9ibd2v,pattern size=6pt,pattern thickness=0.75pt,pattern radius=0pt, pattern color={rgb, 255:red, 208; green, 2; blue, 27}][line width=1.5]  (531.66,215.69) .. controls (528.5,213) and (525.99,209.5) .. (524.45,205.48) -- (546.8,196) -- cycle ;
    %Shape: Pie [id:dp68790328086255] 
    \draw  [color={rgb, 255:red, 208; green, 2; blue, 27 }  ,draw opacity=1 ][pattern=_2hninm6s4,pattern size=6pt,pattern thickness=0.75pt,pattern radius=0pt, pattern color={rgb, 255:red, 208; green, 2; blue, 27}][line width=1.5]  (561.71,175.79) .. controls (564.77,178.34) and (567.25,181.65) .. (568.89,185.47) -- (546.79,196.25) -- cycle ;
    %Curve Lines [id:da48542030785038226] 
    \draw    (535.6,142.3) .. controls (558.37,192.79) and (568.4,127.63) .. (563.16,181.34) ;
    \draw [shift={(563,183)}, rotate = 275.64] [color={rgb, 255:red, 0; green, 0; blue, 0 }  ][line width=0.75]    (10.93,-3.29) .. controls (6.95,-1.4) and (3.31,-0.3) .. (0,0) .. controls (3.31,0.3) and (6.95,1.4) .. (10.93,3.29)   ;
    %Curve Lines [id:da6618733890172424] 
    \draw    (538.6,256.3) .. controls (544.74,225.31) and (523.24,257.35) .. (533.29,206.86) ;
    \draw [shift={(533.6,205.3)}, rotate = 101.58] [color={rgb, 255:red, 0; green, 0; blue, 0 }  ][line width=0.75]    (10.93,-3.29) .. controls (6.95,-1.4) and (3.31,-0.3) .. (0,0) .. controls (3.31,0.3) and (6.95,1.4) .. (10.93,3.29)   ;

    % Text Node
    \draw (556,84) node [anchor=north west][inner sep=0.75pt]  [color={rgb, 255:red, 155; green, 155; blue, 155 }  ,opacity=1 ] [align=left] {$\displaystyle k_{x}$};
    % Text Node
    \draw (656,200) node [anchor=north west][inner sep=0.75pt]  [color={rgb, 255:red, 155; green, 155; blue, 155 }  ,opacity=1 ] [align=left] {$\displaystyle k_{y}$};
    % Text Node
    \draw (406,242) node [anchor=north west][inner sep=0.75pt]   [align=left] {$\displaystyle \eta _{\theta }^{-}$};
    % Text Node
    \draw (674,134) node [anchor=north west][inner sep=0.75pt]   [align=left] {$\displaystyle \eta _{\theta }^{+}$};
    % Text Node
    \draw (470,158) node [anchor=north west][inner sep=0.75pt]   [align=left] {$\displaystyle \lambda _{\theta }^{-}$};
    % Text Node
    \draw (586,220) node [anchor=north west][inner sep=0.75pt]   [align=left] {$\displaystyle \lambda _{\theta }^{+}$};
    % Text Node
    \draw (277,89) node [anchor=north west][inner sep=0.75pt]  [color={rgb, 255:red, 155; green, 155; blue, 155 }  ,opacity=1 ] [align=left] {$\displaystyle k_{x}$};
    % Text Node
    \draw (378,204) node [anchor=north west][inner sep=0.75pt]  [color={rgb, 255:red, 155; green, 155; blue, 155 }  ,opacity=1 ] [align=left] {$\displaystyle k_{y}$};
    % Text Node
    \draw (166,98) node [anchor=north west][inner sep=0.75pt]   [align=left] {$\displaystyle \eta _{\theta }$};
    % Text Node
    \draw (294,292) node [anchor=north west][inner sep=0.75pt]   [align=left] {$\displaystyle \eta _{\theta }^{\prime}$};
    % Text Node
    \draw (540.6,259.3) node [anchor=north west][inner sep=0.75pt]   [align=left] {$\displaystyle \ell _{\theta }^{-}$};
    % Text Node
    \draw (514,126) node [anchor=north west][inner sep=0.75pt]   [align=left] {$\displaystyle \ell _{\theta }^{+}$};
    \end{tikzpicture}
        \caption{Partition of wavevector space to measure the various contributions to the flux of enstrophy into $\eta_{\theta}^{+}$.}
    \label{fig:wave_vec_partition}
\end{figure}
 

Let $\alpha_{\bfkn{1}, \bfkn{2}}^{\bfkn{3}} = \left(k_{1 x} k_{2 y}-k_{2 x} k_{1 y}\right)\left(\frac{1}{\left|\mathbf{k}_{1}\right|^{2}}-\frac{1}{\left|\mathbf{k}_{2}\right|^{2}}\right) a_{\mathbf{k}_{1}} a_{\mathbf{k}_{2}} a_{\mathbf{k}_{3}}$, then we define the following enstrophy flux collective phase for $\mathcal{C}_\theta$

\begin{align}
R_{\mathcal{C}_\theta}\e^{\ii\Phi_{\mathcal{C}_{\theta}}} &= \sum_{\substack{\bfkn{3} \in \mathcal{C}_{\theta} \\ \bfkn{1},  \bfkn{2} \in \mathcal{C}_{\theta}^{'} \\ \bfkn{1} + \bfkn{2} = \bfkn{3}}} \alphakkk{{1}}{2}{3}\e^{\ii \triad{\bfkn{1}}{\bfkn{2}}{\bfkn{3}}} + \sum_{\substack{\bfkn{1}, \bfkn{2} \in \mathcal{C}_{\theta} \\ \bfkn{3} \in \mathcal{C}_{\theta}^{'} \\ \bfkn{1} + \bfkn{2} = \bfkn{3}}} \alphakkk{{1}}{2}{3}\e^{\ii \triad{\bfkn{1}}{\bfkn{2}}{\bfkn{3}}} 
\label{eq:kuramoto_order_param}
\end{align}
where we restrict the summation in the wavevectors to be the right half plane i.e., $k_y > 0$, therefore we only consider the flux into/out of $\mathcal{C}_{\theta}^{+}$. Furthermore the set $\mathcal{C}_{\theta}^{'} \equiv {\mathcal{C}_{\theta}^{+}}^{'}$ can be written as

\begin{align}
	{\mathcal{C}_{\theta}^{+}}^{'} = \bigcup_{\tilde{\theta} \neq \theta} \mathcal{C}_{\tilde{\theta}} \bigcup \mathcal{C}_{\theta}^{L}
\end{align}
where $\mathcal{C}_{\theta}^{L}$ is the lower part of the sector $S_{\theta}$.

Therefore the first term in (\ref{eq:kuramoto_order_param}) becomes

\begin{align}
	\sum_{\substack{\bfkn{3} \in \mathcal{C}_{\theta} \\ \bfkn{1},  \bfkn{2} \in \mathcal{C}_{\theta}^{'} \\ \bfkn{1} + \bfkn{2} = \bfkn{3}}} \alphakkk{{1}}{2}{3}\e^{\ii \triad{\bfkn{1}}{\bfkn{2}}{\bfkn{3}}} &= \sum_{\substack{\bfkn{3} \in \mathcal{C}_{\theta}^{+} \\ \bfkn{1},  \bfkn{2} \in \bigcup_{\tilde{\theta} \neq \theta} \mathcal{C}_{\tilde{\theta}} \bigcup \mathcal{C}_{\theta}^{L} \\ \bfkn{1} + \bfkn{2} = \bfkn{3}}} \alphakkk{{1}}{2}{3}\e^{\ii \triad{\bfkn{1}}{\bfkn{2}}{\bfkn{3}}} \notag \\
	&= \sum_{\tilde{\theta} \neq \theta} \sum_{\substack{\bfkn{3} \in \mathcal{C}_{\theta}^{+} \\ \bfkn{1} \in \mathcal{C}_{\tilde{\theta}} \\ \bfkn{2} \notin \mathcal{C}_{\theta}^{+} \\ \bfkn{1} + \bfkn{2} = \bfkn{3}}} \alphakkk{{1}}{2}{3}\e^{\ii \triad{\bfkn{1}}{\bfkn{2}}{\bfkn{3}}} + \sum_{\substack{\bfkn{3} \in \mathcal{C}_{\theta}^{+} \\ \bfkn{1} \in \mathcal{C}_{\theta}^{L} \\ \bfkn{2} \notin \mathcal{C}_{\theta}^{+} \\ \bfkn{1} + \bfkn{2} = \bfkn{3}}} \alphakkk{{1}}{2}{3}\e^{\ii \triad{\bfkn{1}}{\bfkn{2}}{\bfkn{3}}}
\end{align}

and the second term in (\ref{eq:kuramoto_order_param}) becomes

\begin{align}
	\sum_{\substack{\bfkn{1}, \bfkn{2} \in \mathcal{C}_{\theta} \\ \bfkn{3} \in \mathcal{C}_{\theta}^{'} \\ \bfkn{1} + \bfkn{2} = \bfkn{3}}} \alphakkk{{1}}{2}{3}\e^{\ii \triad{\bfkn{1}}{\bfkn{2}}{\bfkn{3}}} &= \sum_{\substack{\bfkn{1}, \bfkn{2} \in \mathcal{C}_{\theta}^{+} \\ \bfkn{3} \in \bigcup_{\tilde{\theta} \neq \theta} \mathcal{C}_{\tilde{\theta}} \bigcup \mathcal{C}_{\theta}^{L} \\ \bfkn{1} + \bfkn{2} = \bfkn{3}}} \alphakkk{{1}}{2}{3}\e^{\ii \triad{\bfkn{1}}{\bfkn{2}}{\bfkn{3}}} \notag \\
	&= \sum_{\tilde{\theta} \neq \theta} \sum_{\substack{\bfkn{1}, \bfkn{2} \in \mathcal{C}_{\theta}^{+} \\ \bfkn{3} \in \mathcal{C}_{\tilde{\theta}} \\ \bfkn{1} + \bfkn{2} = \bfkn{3}}} \alphakkk{{1}}{2}{3}\e^{\ii \triad{\bfkn{1}}{\bfkn{2}}{\bfkn{3}}} + \sum_{\substack{\bfkn{1}, \bfkn{2} \in \mathcal{C}_{\theta}^{+} \\ \bfkn{3} \in \mathcal{C}_{\theta}^{L} \\ \bfkn{1} + \bfkn{2} = \bfkn{3}}} \alphakkk{{1}}{2}{3}\e^{\ii \triad{\bfkn{1}}{\bfkn{2}}{\bfkn{3}}}
\end{align}

Therefore to obtian the flux into/out of $\mathcal{C}_{\theta}^{+}$ we have

\begin{align}
	\Pi_{\mathcal{C}_{\theta}^{+}} = \sum_{\tilde{\theta} \neq \theta} \left( \sum_{\substack{\bfkn{3} \in \mathcal{C}_{\theta}^{+} \\ \bfkn{1} \in \mathcal{C}_{\tilde{\theta}} \\ \bfkn{2} \notin \mathcal{C}_{\theta}^{+} \\ \bfkn{1} + \bfkn{2} = \bfkn{3}}} \alphakkk{{1}}{2}{3}\e^{\ii \triad{\bfkn{1}}{\bfkn{2}}{\bfkn{3}}} + \sum_{\substack{\bfkn{1}, \bfkn{2} \in \mathcal{C}_{\theta}^{+} \\ \bfkn{3} \in \mathcal{\tilde{\theta}} \\ \bfkn{1} + \bfkn{2} = \bfkn{3}}} \alphakkk{{1}}{2}{3}\e^{\ii \triad{\bfkn{1}}{\bfkn{2}}{\bfkn{3}}} \right) +  \sum_{\substack{\bfkn{3} \in \mathcal{C}_{\theta}^{+} \\ \bfkn{1} \in \mathcal{C}_{\theta}^{L} \\ \bfkn{2} \notin \mathcal{C}_{\theta}^{+} \\ \bfkn{1} + \bfkn{2} = \bfkn{3}}} \alphakkk{{1}}{2}{3}\e^{\ii \triad{\bfkn{1}}{\bfkn{2}}{\bfkn{3}}} + \sum_{\substack{\bfkn{1}, \bfkn{2} \in \mathcal{C}_{\theta}^{+} \\ \bfkn{3} \in \mathcal{C}_{\theta}^{L} \\ \bfkn{1} + \bfkn{2} = \bfkn{3}}} \alphakkk{{1}}{2}{3}\e^{\ii \triad{\bfkn{1}}{\bfkn{2}}{\bfkn{3}}}
\end{align}


which can be loosely interpretted as sum of the contributions to the flux in 1D (along the sector $S_\theta$) and contributions to the flux in 2D which is a function of the angles $\theta$ (denoting the sector that $\bfkn{3}$ lies in) and $\tilde{\theta}$ (the sector that $\bfkn{1}$ lies in)
	
\begin{align}
	\Pi_{\mathcal{C}_{\theta}^{+}} = \sum_{\tilde{\theta} \neq \theta} \Pi_{\text{2D}}(\theta, \tilde{\theta}) + \Pi_{\text{1D}}
\end{align}
and this term can be validated against the flux computed using the nonlinear term of the equations of motion where you consider only the flux contained in the modes in $\mathcal{C}_{\theta}^{+}$

\begin{align}
	\Pi_{\mathcal{C}_{\theta}^{+}} \simeq \der{}{t} \sum_{\bfk \in \mathcal{C}_{\theta}^{+}} |\omega_{\bfk}|^2
\end{align}



\section{Extending To Inlcude Forcing and Inertial Scales}

The 1D and 2D contribution terms can be furhter broken down to specifically include the contribution to the flux into/out of $\eta_{\theta}^{+}$ from the forcing scales $\ell$ and inertial scales $\lambda$

\begin{align}
  \Pi_{\text{1D}}(\theta, \ell, \lambda) = \sum_{\substack{\bfkn{3} \in \eta_{\theta}^{+} \\ \bfkn{1} \in \ell_{\theta}^{-} \bigcup \ell_{\theta}^{+} \\ \bfkn{2} \notin \eta_{\theta}^{+} \\ \bfkn{1} + \bfkn{2} = \bfkn{3}}} +  \sum_{\substack{\bfkn{3} \in \eta_{\theta}^{+} \\ \bfkn{1} \in \lambda_{\theta}^{-} \bigcup \lambda_{\theta}^{+} \\ \bfkn{2} \notin \eta_{\theta}^{+} \\ \bfkn{1} + \bfkn{2} = \bfkn{3}}} + \sum_{\substack{\bfkn{1}, \bfkn{2} \in \eta_{\theta}^{+} \\ \bfkn{3} \in \ell_{\theta}^{-} \bigcup \ell_{\theta}^{+} \\ \bfkn{1} + \bfkn{2} = \bfkn{3}}} + \sum_{\substack{\bfkn{1}, \bfkn{2} \in \eta_{\theta}^{+} \\ \bfkn{3} \in \lambda_{\theta}^{-} \bigcup \lambda_{\theta}^{+} \\ \bfkn{1} + \bfkn{2} = \bfkn{3}}}
\end{align}


\end{document}